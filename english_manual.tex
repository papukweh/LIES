\documentclass[a4paper, 11pt]{article}
\usepackage[utf8]{inputenc}

\title{\textbf{LIES} \\ \small{An accusation card game}}
\author{Gabriel S. Matheus L. Nathalia B.}

\begin{document}

\maketitle

\newpage

\newpage

\section*{INTRODUCTION}

%sobre o jogo, pegar parte do readme.md do git

\section*{CONTENT}

	\begin{itemize}

	\item 12 Character cards
		\begin{itemize}

		\item Detective
		\item Policeman
		\item Spy
		\item Secret Agent
		\item Criminal
		\item Smuggler
		\item Medic
		\item Paramedic
		\item Medium
		\item Oracle
		\item Gambler
		\item Athlete

		\end{itemize}
		
	\item 12 Weapon cards
		\begin{itemize}

		\item 9mm Pistol
		\item Colt .38
		\item Darts
		\item Knife
		\item Switchblade
		\item Ceremonial dagger
		\item Poison
		\item Overdose
		\item Gas
		\item Crystal ball
		\item Brass knuckles
		\item Wooden bat	

		\end{itemize}

	\item 12 Location cards
		\begin{itemize}

		\item Court
		\item Police station
		\item Embassy
		\item Ministry
		\item Bar
		\item Abandoned warehouse
		\item Clinic
		\item Hospital
		\item Museum
		\item Temple
		\item Casino
		\item Stadium

		\end{itemize}

	\item 24 Clue cards
		\begin{itemize}

		\item 6 common Motive Clues
		\item 6 common Trace of Presence Clues
		\item 3 uncommon Material Evidence Clues
		\item 3 uncommon Possession Record Clues
		\item 3 uncommon Criminal Record Clues
		\item 1 rare "Move the body" Clue
		\item 1 rare "Postmortem wound" Clue
		\item 1 rare "Time of death" Clue

		\end{itemize}

	\item 4 Wound cards
		\begin{itemize}

		\item Slash wound
		\item Pierce wound
		\item Chemical wound
		\item Blunt wound

		\end{itemize}

	\item 8 Trust markers

	\end{itemize}

\section*{INSTRUCTIONS}

\vspace{5mm}

%subsections com preparo, curso de jogo, cartas, habilidades, mecânicas (nessa ordem?)
%fazer um appendix pra tabelas ou já jogar aqui mesmo?

\subsection*{PREPARING}

	First, it is necessary to prepare the game deck according to the number of players. To that extent, organize the non-Character cards in four stacks, according to their type - Location, Weapon, Clue and Wound.

	Distribute one Character card for each player. They contain the name of the character, the location they can usually be found at and the weapons they generally use. Remove from the Location and Weapon stacks all cards that aren't related to any of the characters in the game.


%colocar um balaozinho de exemplo ou algo do tipo?
	
	Next, randomly choose a Location card, a Weapon card and a Wound card from their respective stacks and place them on the table. Pick also a Character card from the remaining ones. Place the Location card on the center of the table, partially overlapped by the Weapon and Character cards chosen (indicating that the victim was found on this location next to this weapon) as well as the Wound card, which should partially overlap the Character card (indicating that the character was found with this type of wound). 

	According to the number of players, build the deck like the following:

%fazer uma tabela fofinha aqui, se é que isso vai continuar assim

	3 players - 

	4 players - 

	5 players - 

	6 players - 

	7 players  - 

	8 players - 

	At last, distribute at random 5 cards from the game deck for each player. Each player should place their Character card in front of them and set their Trust marker to 7. The game is now ready to begin.

\subsection*{GAME COURSE}

	A match of LIES is made of rounds, and each rounds is made of turns. The objective of each round is to incriminate a player, who will then participate in the next round as a Witness (made clear by flipping your Character card). This proccess is repeated until there are only 3 players left. On the last round, the player who makes the final argument to incriminate their opponent will be declared the winner.

	On each round, the turns progress as so:

	\begin{itemize}
		\item If this is the first round, a player is randomly chosen to make the first accusation. This player will be called the active player untill the end of the turn.

		\item Otherwise, the player who made the final argument to incriminate someone on the previous round gets to make the first accusation. This player will also be called the active player.
	\end{itemize}

	\begin{enumerate}
		\item The active player makes their accusation, using as many cards as they wishes, or basing it upon cards already on the table. The accusation must be made of only one argument.

		\item The other players (including the accused) may answer the accusations with two moves:
			\begin{itemize}
				\item Bamboozle: use their cards to counter the accusation, effectively nullifying it.

				\item Shenanigan: use their cards to support the accusation, making it stronger.
			\end{itemize}

		\item The cards are then stacked in the order they were played, and should be removed from the stack according to the validity of the Bamboozles and Shenanigans involved. These are then put on the table or stay with the accused player.

		\item The accused player loses Trust according to the argument's strength.

		\item The turn is over. All the players who played cards in this turn should draw one card from the deck. The active player for the next turn is either:
			 \begin{itemize}
				\item The accused

				\item The player who succeeded in their Bamboozle

				\item The player who succeeded in their Shenanigan
			\end{itemize}

	\end{enumerate}

\subsection*{TRUST}

\subsection*{CARDS}

%aqui vão ter imagens bonitinhas de cada carta

	There are 5 types of cards on the game: Character cards, Location cards, Wound cards and Clue cards.


	\begin{itemize}
		\item Character Cards: these cards are distributed before the game begins. They represent the character the player will impersonate during the game and indicate weapons and locations related to them. On the back, each character card represents a Witness. When your character is incriminated, you should flip your card to indicate you're a Witness for the next turns.

		\item Location Cards: these cards indicate physical spaces where bodies, weapons, material evidences and traces are found. Each Location card also indicates the characters who are commonly found in the area. When played, they should be placed on adjacent to existing Locations in game, in order to build a map of the crime scene(s).

		\item Wound Cards: these cards indicate the type of wound found on the victim. A victim may have at maximum two types of wounds active. When a Wound card is played, it should be placed partially on top of the victim card.

		\item Weapon Cards: these cards represent weapons that were found on the crime scene. Each Weapon card indicates which character usually uses them. When played, they should be placed partially on top of a Location card already in the game, indicating the place where the weapon was found.
	\end{itemize}

	\subsubsection*{Clues}

		Clue cards are a wider variety of cards, and the main tool used to accuse other players and make arguments during a debate. Clues are a way to abstract evidence and motives which come up during the game, which means the motive behind a "Motive" clue, for example, isn't pre-defined: it depends upon the creativity of the player who plays it and the current state of the crime scene. They are used to make arguments based on cards already in game, strengthen the arguments of other cards, make a weak argument (when played alone) or even to contradict and nullify other players' accusations. The Clue cards are subdivided like this:

		\begin{itemize}
			\item Motive
			\item Material Evidence
			\item Trace of Presence
			\item Possession Record
			\item Criminal Record
			\item Move the Body
			\item Postmortem Wound
			\item Time of Death
		\end{itemize}

\subsection*{ARGUMENTS}

\end{document}
