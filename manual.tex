\documentclass[a4paper, 11pt]{article}
\usepackage[utf8]{inputenc}

\title{\textbf{LIES} \\ \small{An accusation card game}}
\author{Gabriel S. Matheus L. Nathalia B.}

\begin{document}

\maketitle

\newpage

\newpage

\section*{INTRODUÇÃO}

%sobre o jogo e contents?

\section*{INSTRUÇÕES}

\vspace{5mm}

%subsections com preparo, curso de jogo, cartas, habilidades, mecânicas (nessa ordem?)
%fazer um appendix pra tabelas ou já jogar aqui mesmo?

\subsection*{PREPARO}

Primeiro, é necessário construir o baralho de jogo em função do número de jogadores/as. Para isso, separe as cartas que não são de personagem em 4 pilhas, de acordo com as categorias de cartas - lugar, arma, pista e ferimento. 

Distribua uma carta de personagem para cada jogador/a. Cada uma delas possui o nome do/a personagem, o lugar onde ele/a geralmente é encontrado e as armas que ele/a geralmente usa. Remova das pilhas de lugar e arma quaisquer cartas que não estejam nas cartas de personagem sorteadas.

%colocar um balaozinho de exemplo ou algo do tipo?

Em seguida, retire aleatoriamente uma carta de lugar, uma de arma e uma de ferimento dos respectivos montes e coloque as na mesa. Retire também uma carta de personagem dentre as que sobraram. Coloque a carta de lugar no centro da mesa, com as cartas de arma e personagem parcialmente sobre ela (indicando que a vítima foi encontrada nesse lugar junto dessa arma) e a carta de ferimento cobrindo parcialmente a de personagem (indicando que o/a personagem foi encontrado com este ferimento).

De acordo com o número de jogadores/as, construa o baralho de jogo da seguinte forma:

%fazer uma tabela fofinha aqui, se é que isso vai continuar assim

3 jogadores - 

4 jogadores - 

5 jogadores - 

6 jogadores - 

7 jogadores  - 

8 jogadores - 

Finalmente, distribua aleatoriamente 5 cartas do baralho de jogo para cada jogador/a. Cada jogador/a coloca sua carta de personagem à sua frente e o jogo está pronto para começar.


\end{document}