\documentclass[a4paper, 11pt]{article}
\usepackage[utf8]{inputenc}

\title{\textbf{LIES} \\ \small{An accusation card game}}
\author{Gabriel S. Matheus L. Nathalia B.}

\begin{document}

\maketitle

\newpage

\newpage

\section*{INTRODUÇÃO}

%sobre o jogo, pegar parte do readme.md do git
%regra de aproximacao do tempo de partida t(n) em minutos: Seja n E N o número de jogadores tq n >=3. t(n) ~= {20 se n = 3; 20*(n-2) se n > 3; 0 se n > 8, pq não pode >:( }

\section*{CONTEÚDO}

	\begin{itemize}

	\item 12 cartas de Personagem
		\begin{itemize}

		\item Detetive
		\item Policial
		\item Espião
		\item Agente Secreto
		\item Criminoso
		\item Contrabandista
		\item Médico
		\item Paramédico
		\item Médium
		\item Oráculo
		\item Apostar
		\item Atleta

		\end{itemize}
		
	\item 12 cartas de Arma
		\begin{itemize}

		\item Pistola 9mm
		\item Colt .38
		\item Dardo
		\item Faca
		\item Canivete
		\item Adaga cerimonial
		\item Veneno
		\item Overdose
		\item Gás
		\item Bola de Cristal
		\item Soco inglês
		\item Bastão de Madeira	

		\end{itemize}

	\item 12 cartas de Local
		\begin{itemize}

		\item Tribunal
		\item Delegacia
		\item Embaixada
		\item Ministério
		\item Bar
		\item Armazém abandonado
		\item Consultório
		\item Hospital
		\item Museu
		\item Templo
		\item Cassino
		\item Estádio

		\end{itemize}

	\item 24 cartas de Pista
		\begin{itemize}

		\item 6 Pistas comuns de Motivo
		\item 6 Pistas comuns de Indício de presença
		\item 3 Pistas incomuns de Evidência Material
		\item 3 Pistas incomuns de Registro de posse
		\item 3 Pistas incomuns de Histórico Criminal
		\item 1 Pista rara "Mover o corpo"
		\item 1 Pista rara "Ferimentos Pós-Morte"
		\item 1 Pista rara "Hora da Morte"

		\end{itemize}

	\item 4 cartas de Ferimento
		\begin{itemize}

		\item Corte
		\item Tiro
		\item Químico
		\item Impacto

		\end{itemize}

	\item 8 marcadores de Confiança

	\end{itemize}

\section*{INSTRUÇÕES}

\vspace{5mm}

%subsections com preparo, curso de jogo, cartas, habilidades, mecânicas (nessa ordem?)
%fazer um appendix pra tabelas ou já jogar aqui mesmo?

\subsection*{PREPARO}

	Primeiro, é necessário construir o baralho de jogo em função do número de jogadores(as). Para isso, separe as cartas que não são de personagem em 4 pilhas, de acordo com as categorias de cartas - lugar, arma, pista e ferimento. 

	Distribua uma carta de personagem para cada jogador(a). Cada uma delas possui o nome da personagem, o lugar onde ele(a) geralmente é encontrado e as armas que ele(a) geralmente usa. Remova das pilhas de lugar e arma quaisquer cartas que não estejam nas cartas de personagem sorteadas.

%colocar um balaozinho de exemplo ou algo do tipo?

	Em seguida, retire aleatoriamente uma carta de lugar, uma de arma e uma de ferimento dos respectivos montes e coloque as na mesa. Retire também uma carta de personagem dentre as que sobraram. Coloque a carta de lugar no centro da mesa, com as cartas de arma e personagem parcialmente sobre ela (indicando que a vítima foi encontrada nesse lugar junto dessa arma) e a carta de ferimento cobrindo parcialmente a de personagem (indicando que o/a personagem foi encontrado com este ferimento).

	De acordo com o número de jogadores(as), construa o baralho de jogo da seguinte forma:

%fazer uma tabela fofinha aqui, se é que isso vai continuar assim
%quero bastante matar isso aqui, nem usamos mais

	3 jogadores - 

	4 jogadores - 

	5 jogadores - 

	6 jogadores - 

	7 jogadores  - 

	8 jogadores - 

	Finalmente, distribua aleatoriamente 5 cartas do baralho de jogo para cada jogador(a). Cada jogador(a) coloca sua carta de personagem à sua frente e coloca seu marcador de confiança em 7. O jogo está pronto para começar.

\subsection*{CURSO DE JOGO}

	Uma partida de LIES é composta por rodadas e cada rodada é composta por turnos. O objetivo de cada rodada é incriminar um jogador, que será então eliminado e participará na próxima jogada apenas como uma testemunha (indicado por virar sua carta de personagem ao contrário). Esse processo se repete até que sobrem apenas 3 jogadores(as). Nessa rodada, o(a) jogador(a) que fizer o argumento final para incriminar seu oponente será declarado(a) vencedor(a).

	Dentro de cada rodada, os turnos prosseguem da seguinte maneira:

	\begin{itemize}
		\item Caso seja a primeira rodada, é escolhido um(a) jogador(a) aleatoriamente para fazer a primeira acusação. Este(a) jogador(a) será chamado(a) de jogador(a) ativo(a) até o fim do turno.

		\item Caso contrário, o(a) jogador(a) que fez o argumento final para incriminar um oponente na jogada anterior fará a primeira acusação. O mesmo vale sobre a nomenclatura de jogador(a) ativo(a).
	\end{itemize}

	\begin{enumerate}
		\item O(a) jogador(a) ativo(a) faz sua acusação, usando quantas cartas desejar e podendo relacioná-las com cartas já jogadas. A acusação deve ser composta apenas por um argumento.

		\item Os(as) outros(as) jogadores(as) (incluindo o(a) acusado(a)) podem responder as acusações com suas cartas das seguintes formas:
			\begin{itemize}
				\item Firula: usar suas cartas para contrariar a acusação, de modo que ela se torne nula.

				\item Lorota: usar suas cartas para apoiar a acusação, de modo que ela se torne mais forte. \textit{(O(A) acusado(a) não pode lorotear a si mesmo(a))}
			\end{itemize}

		\item As cartas são empilhadas na ordem que foram jogadas. Elas vão sendo desempilhadas conforme os(as) jogadores(as) analizam quais acusações, lorotas e firulas acabaram sendo válidas. Estas últimas são colocadas ou em campo ou no jogador acusado.

		\item O(a) jogador(a) acusado(a) perde confiança de acordo com a força do argumento. \\* Caso algum jogador tenha zero pontos de confiança, a rodada acaba, com esse(a) jogador(a) se tornando uma testemunha para as próximas rodadas.

		\item O turno acaba. Todos(as) os(as) jogadores(as) que jogaram cartas nesse turno compram uma carta. O(a) jogador(a) ativo(a) do próximo turno é:
			\begin{itemize}
				\item Aquele(a) que foi acusado(a) neste turno.

				\item O(a) jogador(a) que conseguiu "Firular" a acusação.

				\item O(a) jogador(a) que mais fortaleceu a acusação "Loroteando".
			\end{itemize}

	\end{enumerate}

	\begin{itemize}
		\item Caso não haja mais como o jogo prosseguir (falta de cartas no baralho, jogadores(as) sem cartas nas mãos, etc) e não haja nenhum(a) jogador(a) com confiança 0, o(a) jogador(a) com menor confiança é eliminado(a). Caso haja mais do que um(a) jogador(a) com menor confiança, os(as) jogadores(as) e testemunhas votam em qual deles(as) será eliminado. Caso a votação seja um empate, o(a) jogador(a) eliminado(a) deve ser decidido(a) aleatoriamente  entre os(as) que tem menor confiança.
	\end{itemize}

\subsection*{CARTAS}

%aqui vão ter imagens bonitinhas de cada carta

	Em LIES, existem 5 tipos de cartas: cartas de personagem, cartas de lugar, cartas de ferimento, cartas de arma e pistas.

	\begin{itemize}
		\item \textbf{Cartas de Personagem} - estas cartas são distribuídas antes do jogo começar. Elas representam o personagem que o jogador será durante o jogo e indicam que armas geralmente estão relacionadas a ele e em que lugar ele geralmente é encontrado. No verso, cada carta de personagem representa uma testemunha. Quando seu personagem for incriminado, você deve virar sua carta de personagem para indicar que se tornou uma testemunha para as próximas rodadas.

		\item \textbf{Cartas de Lugar} - estas cartas indicam lugares físicos onde são encontrados corpos, armas, evidências materiais e outros rastros. Cada carta de lugar indica também o personagem que geralmente é encontrado lá. Quando elas são jogadas, devem ser colocadas em posições adjacentes aos lugares que já foram jogados, construindo um mapa das cenas do crime.

		\item \textbf{Cartas de ferimento} - estas cartas indicam o tipo de ferimento que foi encontrado na vítima. Uma vítima pode ter um máximo de dois tipos de ferimento ao mesmo tempo. Quando uma carta de ferimento é jogada, ela deve ser colocada parcialmente sobre a vítima.

		\item \textbf{Cartas de arma} - estas cartas representam armas que foram encontradas na cena do crime. Cada carta de arma indica que personagens geralmente usam essas armas. Quando elas são jogadas, devem ser colocadas parcialmente sobre alguma carta de lugar já jogada, indicando que foi lá que ela foi encontrada.
	\end{itemize}

	\subsubsection*{Pistas}

		Pistas são uma categoria mais ampla de cartas, sendo a principal forma de acusar outros(as) jogadores(as) e argumentar durante um debate. Pistas abstraem as evidências e motivos que surgem durante o jogo, ou seja, o motivo por trás de uma pista de "Motivo" não é pré-definido, mas sim depende da criatividade do(a) jogador(a) que joga a carta e de como a cena do crime foi construída até o momento. Elas são usadas para fazer argumentos com base nas cartas que já foram jogadas, fortalecer argumentos compostos por outras cartas, fazer argumentos fracos ao serem jogadas individualmente ou até contradizer e anular acusações de outros(as) jogadores(as). 
		
		Após o final da argumentação, as pistas jogadas podem ou ser descartadas ou ser colocadas em algum jogador, lugar ou arma para que sua relevância seja lembrada para argumentos futuros. Caso sejam firuladas, elas são descartadas.
		
		As pistas se dividem nos seguintes tipos:

		\begin{itemize}
			\item \textbf{Motivo} - Abstrai a existência de um motivo que um personagem tenha para ou matar a vítima ou que vá contra ele querer a vítima morta. Caso o argumento que continha o Motivo não seja firulado, a carta de Motivo é colocada junto ao personagem a que ela faz referência (o(a) acusado(a) de possuir esse motivo, por exemplo).
			\item \textbf{Evidência Material} -  Evidências de que determinado personagem esteve no local do crime ou de que alguma arma tenha sido manipulada, causando ferimentos que geralmente não causaria. Caso o argumento que continha a Evidência Material não seja firulado, ele é colocado sobre o lugar ou arma a que faz referência.
			\item \textbf{Indício de Presença} - Representa a existência de uma pessoa ou câmeras que tenham avistado determinado personagem, em algum dos locais postos em mesa, próximo a hora do crime. Pode ser tanto usado para indicar que o personagem estava próximo ao local do assassinato quanto estava longe. Caso não seja firulado, é colocado ou no lugar ou ao lado do personagem a que faz referência.
			\item \textbf{Registro de Posse} - Indica que uma arma, objeto ou local em mesa pertence a certo personagem de jogo, seja representando registros de compra ou similares. Caso não seja firulado, é colocado sobre a arma ou ao lado do personagem a que faz referência.
			\item \textbf{Histórico Criminal} - Registros, dados e documentos que mostram o antecendente criminal de um personagem do jogo. Provavelmente contém denuncias da vitima contra o personagem ou ordens de restrição que impediriam ele de se aproximar da vitima. Caso não seja firulado, é colocado ao lado do personagem a que faz referência.
			\item \textbf{Corpo Movido} - Rastros de sangue, da vitima ter sido arrastada e similares. Basicamente, traços que indicam que o corpo foi movido de um local adjacente da mesa. Em termos de jogo, altera o local de morte da vitima. Essa carta sempre vai para a pilha de descarte, visto que seu efeito é indicado por trocar de lugar o corpo da vítima.
			\item \textbf{Ferimento Pós-Morte} - Indícios que determinado ferimento foi realizado após a morte da vitima. Permite que o jogador remova um ou dois ferimentos do corpo. Essa carta sempre vai para a pilha de descarte, visto que seu efeito é indicado pela remoção das cartas de ferimento do carta da vítima.
			\item \textbf{Hora da Morte} - Representam manipulações no corpo que alteraram o horário esperado da morte, revelando um erro na hora da morte original. Em termos de jogo, pode ser usada para cancelar diversos argumentos de outras cartas como indício de presença e similares. Quando usada, deve-se verificar os argumentos passados e remover da mesa qualquer carta que foi anulada por essa, devido a inconsistências na história construída até o momento.
		\end{itemize}

\subsection*{ARGUMENTOS E CONFIANÇA}

	Confiança é o valor que representa a inocência de um(a) determinado(a) jogador(a). O valor inicial de confiança de cada jogador(a) é 7, que decai conforme ele(a) é acusado(a). Caso chegue a 0 ou menos, o(a) jogador(a) não poderá acusar mais ninguém e, consequentemente, perderá o jogo. Cada jogador(a) possui um marcador de confiança, que deve estar visível para todos os(as) outros(as) jogadores(as) e testemunhas. Testemunhas não possuem marcador de confiança.

	Um(a) jogador(a) perde confiança quando é acusado(a) com sucesso, por meio de um argumento feito por outro(a) jogador(a). Para argumentar, um(a) jogador(a) pode jogar qualquer número de cartas de sua mão e combiná-las com cartas e informações que já estão em jogo. Entretanto, independente do número de cartas jogadas, um argumento sempre será "forte" (subtrai 2 pontos de confiança do(a) jogador(a) acusado(a)) ou "fraco" (subtrai 1 ponto de confiança do(a) jogador(a) acusado(a)). Isso significa que um argumento forte construído com quatro cartas tem poder igual a um construído com duas - porém o de quatro cartas talvez seja mais difícil de enfraquecer com firulas.

%como determinar força de argumentos

	Para firular ou lorotear o argumento com outro, um(a) jogador(a) deve esperar o(a) jogador(a) que está falando no momento terminar de argumentar e de jogar suas cartas. Após a acusação inicial do turno, o(a) jogador(a) que se voluntariar primeiro falando "FIRULA!" ou "LOROTA!" começará seu argumento.

	É importante notar que o argumento que é feito como Firula deve necessariamente contradizer a acusação atual. Caso o(a) jogador(a) deseje apenas acusar seu(a) acusador(a), ele(a) deve aceitar a acusação feita contra si e então, em seu turno, fazer uma nova acusação.

%exemplo disso plis

\end{document}
