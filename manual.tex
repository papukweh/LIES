\documentclass[a4paper, 11pt]{article}
\usepackage[utf8]{inputenc}

\title{\textbf{LIES} \\ \small{An accusation card game}}
\author{Gabriel S. Matheus L. Nathalia B.}

\begin{document}

\maketitle

\newpage

\newpage

\section*{INTRODUÇÃO}

%sobre o jogo, pegar parte do readme.md do git

\section*{CONTEÚDO}

	\begin{itemize}

	\item 12 cartas de Personagem
		\begin{itemize}

		\item Detetive
		\item Policial
		\item Espião
		\item Agente Secreto
		\item Criminoso
		\item Contrabandista
		\item Médico
		\item Paramédico
		\item Médium
		\item Oráculo
		\item Apostar
		\item Atleta

		\end{itemize}
		
	\item 12 cartas de Arma
		\begin{itemize}

		\item Pistola 9mm
		\item Colt .38
		\item Dardo
		\item Faca
		\item Canivete
		\item Adaga cerimonial
		\item Veneno
		\item Overdose
		\item Gás
		\item Bola de Cristal
		\item Soco inglês
		\item Bastão de Madeira	

		\end{itemize}

	\item 12 cartas de Local
		\begin{itemize}

		\item Tribunal
		\item Delegacia
		\item Embaixada
		\item Ministério
		\item Bar
		\item Armazém abandonado
		\item Consultório
		\item Hospital
		\item Museu
		\item Templo
		\item Cassino
		\item Estádio

		\end{itemize}

	\item 24 cartas de Pista
		\begin{itemize}

		\item 6 Pistas comuns de Motivo
		\item 6 Pistas comuns de Indício de presença
		\item 3 Pistas incomuns de Evidência Material
		\item 3 Pistas incomuns de Registro de posse
		\item 3 Pistas incomuns de Histórico Criminal
		\item 1 Pista rara "Mover o corpo"
		\item 1 Pista rara "Ferimentos Pós-Morte"
		\item 1 Pista rara "Hora da Morte"

		\end{itemize}

	\item 4 cartas de Ferimento
		\begin{itemize}

		\item Corte
		\item Tiro
		\item Químico
		\item Impacto

		\end{itemize}

	\item 8 marcadores de Confiança

	\end{itemize}

\section*{INSTRUÇÕES}

\vspace{5mm}

%subsections com preparo, curso de jogo, cartas, habilidades, mecânicas (nessa ordem?)
%fazer um appendix pra tabelas ou já jogar aqui mesmo?

\subsection*{PREPARO}

	Primeiro, é necessário construir o baralho de jogo em função do número de jogadores(as). Para isso, separe as cartas que não são de personagem em 4 pilhas, de acordo com as categorias de cartas - lugar, arma, pista e ferimento. 

	Distribua uma carta de personagem para cada jogador(a). Cada uma delas possui o nome da personagem, o lugar onde ele(a) geralmente é encontrado e as armas que ele(a) geralmente usa. Remova das pilhas de lugar e arma quaisquer cartas que não estejam nas cartas de personagem sorteadas.

%colocar um balaozinho de exemplo ou algo do tipo?

	Em seguida, retire aleatoriamente uma carta de lugar, uma de arma e uma de ferimento dos respectivos montes e coloque as na mesa. Retire também uma carta de personagem dentre as que sobraram. Coloque a carta de lugar no centro da mesa, com as cartas de arma e personagem parcialmente sobre ela (indicando que a vítima foi encontrada nesse lugar junto dessa arma) e a carta de ferimento cobrindo parcialmente a de personagem (indicando que o/a personagem foi encontrado com este ferimento).

	De acordo com o número de jogadores(as), construa o baralho de jogo da seguinte forma:

%fazer uma tabela fofinha aqui, se é que isso vai continuar assim

	3 jogadores - 

	4 jogadores - 

	5 jogadores - 

	6 jogadores - 

	7 jogadores  - 

	8 jogadores - 

	Finalmente, distribua aleatoriamente 5 cartas do baralho de jogo para cada jogador(a). Cada jogador(a) coloca sua carta de personagem à sua frente e coloca seu marcador de confiança em 7. O jogo está pronto para começar.

\subsection*{CURSO DE JOGO}

	Uma partida de LIES é composta por rodadas e cada rodada é composta por turnos. O objetivo de cada rodada é incriminar um jogador, que será então eliminado e participará na próxima jogada apenas como uma testemunha (indicado por virar sua carta de personagem ao contrário). Esse processo se repete até que sobrem apenas 3 jogadores(as). Nessa rodada, o(a) jogador(a) que fizer o argumento final para incriminar seu oponente será declarado vencedor(a).

	Dentro de cada rodada, os turnos prosseguem da seguinte maneira:

	\begin{itemize}
		\item Caso seja a primeira rodada, é escolhido um(a) jogador(a) aleatoriamente para fazer a primeira acusação. Este(a) jogador(a) será chamado(a) de jogador(a) ativo(a) até o fim do turno.

		\item Caso contrário, o(a) jogador(a) que fez o argumento final para incriminar um oponente na jogada anterior fará a primeira acusação. O mesmo vale sobre a nomenclatura de jogador(a) ativo(a).
	\end{itemize}

	\begin{enumerate}
		\item O(a) jogador(a) ativo(a) faz sua acusação, usando quantas cartas desejar e podendo relacioná-las com cartas já jogadas. A acusação deve ser composta apenas por um argumento.

		\item Os(as) outros(as) jogadores(as) (incluindo o(a)) podem responder as acusações com suas cartas das seguintes formas:
			\begin{itemize}
				\item Firula: usar suas cartas para contrariar a acusação, de modo que ela se torne nula.

				\item Lorota: usar suas cartas para apoiar a acusação, de modo que ela se torne mais forte.
			\end{itemize}

		\item As cartas são empilhadas na ordem que foram jogadas. Elas vão sendo desempilhadas conforme os(as) jogadores(as) analizam quais acusações, lorotas e firulas acabaram sendo válidas. Estas últimas são colocadas ou em campo ou no jogador acusado.

		\item O(a) jogador(a) acusado(a) perde confiança de acordo com a força do argumento.

		\item O turno acaba. Todos(as) os(as) jogadores(as) que jogaram cartas nesse turno compram uma carta. O(a) jogador(a) ativo(a) do próximo turno é:
			 \begin{itemize}
				\item Aquele(a) que foi acusado(a) neste turno.

				\item O(a) jogador(a) que conseguiu "Firular" a acusação.

				\item O(a) jogador que mais fortaleceu a acusação "Loroteando".
			\end{itemize}

	\end{enumerate}

\end{document}
